\chapter{Appendix1}

\section*{Input Format of the Map}

A map is inputted in form of a scenario file. Here is a sample scenario file.
\begin{verbatim}
 
26
-3 0
0 -8
7 -8
4 -4
4 -2
5 -2
7 -3
7 -1
6 0
5 -1
4 -1
3 -3
3 -5
2 -5
2 3
-2 3
-4 6
-4 8
-3 8

-1 7
-1 9

-2 10
-3 9
-4 9
-5 7
-5 5
8
-4 8
-3 8
-1 7
-1 9
-2 10
-3 9
-4 9
-4.5 8



-2 8

\end{verbatim}

\noindent The first line, 26 denotes the number of vertices in the polygon. \\
The next 26 lines contains 2 tuples representing the (x,y) coordinates of each point. \\
The next line containing 8, denotes the number of vertices in the visibility polygon. \\
Then 8 lines follow containing the coordinates of the visibility polygon. There is no absolute meaning to these 
coordinates, in the sense they are just a means to depict the shape of the visibility polygon.\\
The last line denotes the position of the robot within the visibility polygon.\\



% ------------------------------------------------------------------------

%%% Local Variables: 
%%% mode: latex
%%% TeX-master: "../thesis"
%%% End: 
