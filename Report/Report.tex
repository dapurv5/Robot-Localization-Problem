\documentclass[a4paper,10pt]{article}
\usepackage[utf8x]{inputenc}
\usepackage{amsmath}
\usepackage{hyperref}

%opening
\title{Robot Localization Simulator}
\author{}

\begin{document}

\maketitle

\begin{abstract}
In this project we explore the various algorithms for the Robot Localization Problem and build a simulator to visualize the results on
various 2D maps.
Robot localization is an important problem in robotics. Stated simply the robot localization problem is as follows. A robot is 
placed at an unknown point inside a simple polygon $ P $. The robot has a map of 
$ P$ and can compute visibility polygon from its current location. The robot must determine its correct 
location inside the polygon $P $ at a minimum cost of travel distance. We implement a ${O(\log^3 n)} $ factor approximation algorithm 
as given by \cite{key1}. We used CGAL \cite{key2}.

\end{abstract}

\newpage

\section{Computing Visibility Polygons}
Visibility polygon is an indispensable component is the hypothesis generation step of the algorithm. Since CGAL had no inbuilt support
 for computing visibility polygons we implemented the following two routines for our purposes.
\begin{enumerate}
 \item Visibility Polygon of a point inside a polygon
 \item Visibility Polygon of an edge of the polygon.
\end{enumerate}

\subsection{Visibility Polygon of a Point Inside a Polygon}

\newpage

\begin{thebibliography}{1}
\bibitem{key1} \emph{A near-tight approximation algorithm for the robot localization problem},
 Koenig, Sven and Mudgal, Apurva and Tovey, Craig A, Proceedings of the Symposium on Discrete Algorithms SODA, 2006.

\bibitem{key2} $<$\url{http://www.cgal.org/}$>$
\end{thebibliography}



\end{document}
